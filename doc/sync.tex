Aby zsynchronizować bazy, należy wykonać zapytanie:

\begin{lstlisting}

    GET http://tomato-cal.herokuapp.com/sync
\end{lstlisting}

Wymagane argumenty:

\begin{itemize}
  \item access\_token - token identyfikujący użytkownika
  \item snapshot - obraz bazy danych na urządzeniu: kolekcja wszystkich list i zadań.
\end{itemize}

Przykład:

\begin{lstlisting}
curl -H 'Content-Type: application/json' -H 'Accept: application/json' -X POST http://tomato-cal.herokuapp.com/sync?access_token=42fF2zhLai3 -d '
  {
    lists: [
      {
        identifier: "rpi_4LK3kll2nza",
        name: "Test list",
        updated_at: "2013-05-30T13:47:41Z",
        tasks: [
          {
            "name" => "Test task",
            "priority"=> 1,
            "updated_at"=>"2015-11-22T15:47:22.768Z",
            "identifier" => 'rpi_0v9Kds2ngi',
            "x"=> 104,
            "y"=> 45,
            "duration" => 3
          }
        ]
      }
    ]
  }'

\end{lstlisting}

Statusy:

\begin{itemize}
  \item 200 - ok
  \item 403 - not authorized
\end{itemize}

Odpowiedzi:

Poprawna (200) - zwracana jest zsynchronizowana tablica list wraz z tablicami zadań danej listy

\begin{lstlisting}
  {
    lists: [
      {
        id: 1,
        identifier: "rpi_4LK3kll2nza",
        name: "Test list",
        created_at: "2013-05-30T13:47:41Z",
        updated_at: "2013-05-30T13:47:41Z",
        tasks: [
          {
            "id" => 1,
            "name" => "Test task",
            "priority"=> 1,
            "created_at"=>"2015-11-22T15:47:22.701Z",
            "updated_at"=>"2015-11-22T15:47:22.768Z",
            "identifier" => task.identifier,
            "x"=> 104,
            "y"=> 45,
            "duration" => 3,
            "list_id" => 1
          }
        ]
      }
    ]
  }
\end{lstlisting}
