Poniżej znajduje się lista akceptowanych przez serwer zapytań od aplikacji, oraz lista zwracanych obiektów w formacie JSON.

Serwer nie powinien zwracać nieprzechwytywanych wyjątków, lecz obiekt JSON z odpowiednim polem statusu oraz odpowiednim opisem błędu:

\subsection{ Forbidden}

Ten błąd zwracany jest w przypadku próby wykonania nieautoryzowanej akcji, np. próby utworzenia listy/zadania bez wcześniejszego zalogowania.

\textbf{Format:}

\begin{lstlisting}
  {
    "message": "You are not authorized to access this resource.",
    "status": 403
  }
\end{lstlisting}

\subsection{ Not found}

Próba odwołania do zasobu nieistniejącego na serwerze.

\textbf{Format:}

\begin{lstlisting}
  {
    "message": "Resource is not found.",
    "status": 404
  }
\end{lstlisting}

\subsection{ Unprocessable entity (422)}

Nieprawidłowy format zapytania, na przykład gdy w parametrach do zadania podamy argumenty bez nazwy obiektu:

\begin{lstlisting}
  {
    "name": "Test 1"
  }
\end{lstlisting}

zamiast:

\begin{lstlisting}
  {
    "task": {
      "name": "Test 1"
    }
  }
\end{lstlisting}

\textbf{Format:}

\begin{lstlisting}
  {
    "message": "Unprocessable entity.",
    "status": 422
  }
\end{lstlisting}

\subsection{ Invalid (unauthorized) (401)}

Autoryzacja jest możliwa, ale się nie udała (przy logowaniu, jeśli podamy nieprawidłowe dane)

\textbf{Format:}

\begin{lstlisting}
  {
    "message": "Incorrect email or password",
    "access_token" : null
    "status": 401
  }
\end{lstlisting}

\subsection{Invalid (406)}

Występuje, gdy nie są spełnione wymogi walidacji, np

\begin{lstlisting}
  {
    "message": "List cannot be saved - please, review the errors below.",
    "list": {
      "errors": {
        "title": [
          "can't be blank",
          "has already been taken, but 'List 2 is available'
        ]
      }
    }
  }
\end{lstlisting}
