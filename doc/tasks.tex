\subsection{Pobranie zadań danej listy:}

Założenie jest takie, że każdy użytkownik ma dostęp wyłącznie do swoich zadań.


\begin{lstlisting}
  GET http://tomato-cal.herokuapp.com/lists/1/tasks?access_token="abc23@klj1309"
\end{lstlisting}


Wymagane argumenty
\begin{itemize}
  \item access\_token - token identyfikujący użytkownika
  \item list\_id - identyfikacja listy
\end{itemize}

Statusy:
\begin{itemize}
  \item 200 - ok
  \item 403 - not authorized
\end{itemize}

\subsubsection{Odpowiedzi:}

Poprawna (200) - zwracana jest tablica zadań

\begin{lstlisting}
  {
   "tasks_count": 9,
    "tasks": [
      {
        "id" => 1,
        "name" => "Test task",
        "priority"=> 1,
        "created_at"=>"2015-11-22T15:47:22.701Z",
        "updated_at"=>"2015-11-22T15:47:22.768Z",
        "identifier" => "serv_llopY8Kz",
        "x"=> 104,
        "y"=> 45,
        "duration" => 3,
        "list_id" => 1
      }
    ]
  }
\end{lstlisting}

\subsection{Pobranie pojedynczego zadania:}

\begin{lstlisting}
    GET http://tomato-cal.herokuapp.com/lists/1/tasks/1?access_token="abc23@klj1309"
\end{lstlisting}

Wymagane argumenty
\begin{itemize}
  \item access\_token - token identyfikujący użytkownika
  \item list\_id - id identyfikujące daną listę
  \item id - id identyfikujące dane zadanie
\end{itemize}

Statusy:
\begin{itemize}
  \item 200 - ok
  \item 403 - not authorized
\end{itemize}

\subsubsection{Odpowiedzi:}

Poprawna (200) - zwracane jest pojedyncze zadanie.

\begin{lstlisting}
  {
    "task" => {
      "id" => 1,
      "name" => "Test task",
      "priority"=> 1,
      "created_at"=>"2015-11-22T15:47:22.701Z",
      "updated_at"=>"2015-11-22T15:47:22.768Z",
      "identifier" => "serv_13llzyI2k",
      "x"=> 104,
      "y"=> 45,
      "duration" => 3,
      "list_id" => 1,
      "list" => {
        "id"=>1,
        "name"=>"Test list",
        "identifier"=>"serv_13llzyI2k",
        "created_at"=>"2015-11-22T15:47:22.701Z",
        "updated_at"=>"2015-11-22T15:47:22.768Z",
        "user_id"=>1
      }
    }
  }
\end{lstlisting}


\subsection{Tworzenie zadania}

\begin{lstlisting}
    POST http://tomato-cal.herokuapp.com/lists/1/tasks.json
\end{lstlisting}

Wymagane argumenty
\begin{itemize}
  \item access\_token - token identyfikujący użytkownika
  \item title - unikalny tytuł
  \item list\_id - identifikator listy
\end{itemize}

Statusy:
\begin{itemize}
  \item 201 - created
  \item 403 - forbidden
  \item 406 - not acceptable - validation error
\end{itemize}

\subsubsection{Odpowiedzi:}

Poprawna (201)
\begin{lstlisting}
  {
    "task" => {
      "id" => 1,
      "name" => "Test task",
      "priority"=> 1,
      "created_at"=>"2015-11-22T15:47:22.701Z",
      "updated_at"=>"2015-11-22T15:47:22.768Z",
      "identifier" => "serv_13llzyI2k",
      "x"=> 104,
      "y"=> 45,
      "duration" => 3,
      "list_id" => 1,
      "list" => {
        "id"=>1,
        "name"=>"Test list",
        "identifier"=>"serv_13llzyI2k",
        "created_at"=>"2015-11-22T15:47:22.701Z",
        "updated_at"=>"2015-11-22T15:47:22.768Z",
        "user_id"=>1
      }
    }
  }
\end{lstlisting}
