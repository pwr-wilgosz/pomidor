Istnieje wiele systemów, które wspomagają prowadzenie i organizację projektów. Są one tworzone głównie z myślą o wykonawcy zleceń, jak i o organizatorach. Zasadą ich działania jest zbieranie informacji od użytkownika dotyczące niezbędnych do wykonania zadań oraz układanie ich w formie planu - terminarza. W użyciu przypominają rozbudowany kalendarz umożliwiający łatwe dodawanie nowych zadań oraz podpowiadającego efektywne zagospodarowanie czasu.

\subsection{Wymagania} % (fold)
\label{sub:wymagania}
System przeznaczony jest dla pojedynczych użytkowników, do użytku własnego. Z tej perspektywy ważnymi, ale nie kluczowymi wyznacznikami urządzenia są:
\begin{enumerate}
	\item zapewnienie maksymalnej niezawodności
	\item wytrzymałość na zniszczenia i warunki ekstremalne
	\item możliwość wykonania kopii zapasowej
\end{enumerate}

Kluczowymi cechami platformy, ze względu na swoje przeznaczenie i środowisko pracy są:
\begin{enumerate}
	\item łatwe zarządzanie danymi
	\item trwałość i pewność zapisu
	\item wysoka dostępność do danych
	\item intuicyjność interfejsu aplikacji
	\item łatwa konfiguracja
	\item płynność działania aplikacji
	\item szybka synchronizacja
\end{enumerate}

Wymagania podstawowe stawiane aplikacji:
\begin{enumerate}
	\item możliwość anulowania operacji
	\item możliwość manipulacji danymi (dodawanie/edycja/usuwanie)
	\item interfejs w języku polskim
\end{enumerate}

Dostępność tłumaczy się jako możliwość podglądu zadań z poziomu aplikacji internetowej jak i aplikacji na platformach mikroprocesorowych zsynchronizowanych z aplikacją internetową.

% section wymagania (end)
\subsection{Ograniczenia} % (fold)
\label{sub:ograniczenia}
Ograniczeniem dla projektu są:
\begin{enumerate}
	\item wymiary
	\subitem platforma mikroprocesorowa powinna być możliwie płaska, oraz posiadać duży ekran dotykowy
	\item zasilanie
	\subitem systemy wbudowane wymagają specjalnego zasilania
	\item dostęp do internetu
	\subitem internet jest niezbędny do dwukierunkowej synchronizacji danych z serwerem
\end{enumerate}

Ograniczeniem dla projektu w przyszłości mogą być:
\begin{enumerate}
	\item przepisy regulujące kwestie związane z używaniem urządzeń elektrycznych w określonym środowisku (np. łazienka)
\end{enumerate}
