\subsection{Logowanie}

Aby zalogować uzytkownika i otrzymać odpowiadający mu klucz dostępu, należy wysłać zapytanie do serwera, przekazując w parametrach email oraz hasło. W przypadku pomyślnego logowania, serwer zwróci kod dostępu, \textit{access\_token}, który od tej pory będzie autoryzował użytkownika.

Każde zapytanie do serwera bez podanego w parametrze klucza dostępu zwróci obiekt JSON odpowiadający błędowi (403) \ref{section:api-errors}


\begin{lstlisting}
  POST http://tomato-cal.herokuapp.com/login.json
\end{lstlisting}

Wymagane parametry:
\begin{itemize}
  \item email
  \item password
\end{itemize}

Statusy odpowiedzi:
\begin{itemize}
  \item 200 - ok
  \item 403 - unauthorised
\end{itemize}

\subsubsection{Odpowiedzi:}

Poprawna (200)
\begin{lstlisting}
  {
      "status": 200,
      "access_token" : "abcdefgh"
  }
\end{lstlisting}

Nieprawidłowe dane logowania (403)
\begin{lstlisting}
  {
    "message" : "Invalid email or password",
    "status": 403
  }
\end{lstlisting}
